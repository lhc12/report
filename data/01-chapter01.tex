\chapter{\label{chap:intr}Introduction}

The study of solid state physics is a long story which even traces back to 18th century. The research of the early stage solid state physics concerned about the macroscopy properties of solids such as hardness, elasticity, resistance. Thanks to these great contribution, the basis of solid state physics was founded. Scientists of that time were trying to establish a theory which describes the change of these macroscopy properties precisely. But without knowing the microscopy structure of matters, it was hard to progress on that way--Some of the phenomenons couldn't get explained with the old methodology. Conductivity was one of the challenge at that time.

People have known that every solid contains electrons since 19th century. And free electrons can form up electrical currents when someone applies an external electric field. Based on this idea, scientists developed the model of free electron gas. This model provides us good insight into thermal conductivities, heat capacity and electrical conductivity of metals. But it doesn't solve the distinction between metals and insulators. For that purpose, physicists tried to involve quantum mechanics into solid state physics since the early age of 20th century. Consequently, it was proved to be a great success. One of the important achievements is the foundation of energy band theory.

Different from the model of free electron gas, energy band theory takes the periodic lattice of solids into account. 

\cite{einstein1905electrodynamics}.
\begin{itemize}
	\item Topological Insulators.
	\item Majorana Fermions.
\end{itemize}

\clearpage

\section{\label{sec:den_fun_the}Topological Insulators}

According to energy band theory, insulators are defined as a type of solids whose Fermi level locates in a band gap. In such a case, electrons are confined locally and cannot form up an electrical current. 

\section{\label{sec:maj_fer}Majorana Fermions}